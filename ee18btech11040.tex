\documentclass[journal,12pt,twocolumn]{IEEEtran}

\usepackage{setspace}
\usepackage{gensymb}
\singlespacing
\usepackage[cmex10]{amsmath}

\usepackage{amsthm}

\usepackage{mathrsfs}
\usepackage{txfonts}
\usepackage{stfloats}
\usepackage{bm}
\usepackage{cite}
\usepackage{cases}
\usepackage{subfig}

\usepackage{longtable}
\usepackage{multirow}

\usepackage{enumitem}
\usepackage{mathtools}
\usepackage{steinmetz}
\usepackage{tikz}
\usepackage{circuitikz}
\usepackage{verbatim}
\usepackage{tfrupee}
\usepackage[breaklinks=true]{hyperref}
\usepackage{graphicx}
\usepackage{tkz-euclide}

\usetikzlibrary{calc,math}
\usepackage{listings}
    \usepackage{color}                                            %%
    \usepackage{array}                                            %%
    \usepackage{longtable}                                        %%
    \usepackage{calc}                                             %%
    \usepackage{multirow}                                         %%
    \usepackage{hhline}                                           %%
    \usepackage{ifthen}                                           %%
    \usepackage{lscape}     
\usepackage{multicol}
\usepackage{chngcntr}

\DeclareMathOperator*{\Res}{Res}

\renewcommand\thesection{\arabic{section}}
\renewcommand\thesubsection{\thesection.\arabic{subsection}}
\renewcommand\thesubsubsection{\thesubsection.\arabic{subsubsection}}

\renewcommand\thesectiondis{\arabic{section}}
\renewcommand\thesubsectiondis{\thesectiondis.\arabic{subsection}}
\renewcommand\thesubsubsectiondis{\thesubsectiondis.\arabic{subsubsection}}


\hyphenation{op-tical net-works semi-conduc-tor}
\def\inputGnumericTable{}                                 %%

\lstset{
%language=C,
frame=single, 
breaklines=true,
columns=fullflexible
}
\begin{document}


\newtheorem{theorem}{Theorem}[section]
\newtheorem{problem}{Problem}
\newtheorem{proposition}{Proposition}[section]
\newtheorem{lemma}{Lemma}[section]
\newtheorem{corollary}[theorem]{Corollary}
\newtheorem{example}{Example}[section]
\newtheorem{definition}[problem]{Definition}

\newcommand{\BEQA}{\begin{eqnarray}}
\newcommand{\EEQA}{\end{eqnarray}}
\newcommand{\define}{\stackrel{\triangle}{=}}
\bibliographystyle{IEEEtran}
\raggedbottom
\setlength{\parindent}{0pt}
\providecommand{\mbf}{\mathbf}
\providecommand{\pr}[1]{\ensuremath{\Pr\left(#1\right)}}
\providecommand{\qfunc}[1]{\ensuremath{Q\left(#1\right)}}
\providecommand{\sbrak}[1]{\ensuremath{{}\left[#1\right]}}
\providecommand{\lsbrak}[1]{\ensuremath{{}\left[#1\right.}}
\providecommand{\rsbrak}[1]{\ensuremath{{}\left.#1\right]}}
\providecommand{\brak}[1]{\ensuremath{\left(#1\right)}}
\providecommand{\lbrak}[1]{\ensuremath{\left(#1\right.}}
\providecommand{\rbrak}[1]{\ensuremath{\left.#1\right)}}
\providecommand{\cbrak}[1]{\ensuremath{\left\{#1\right\}}}
\providecommand{\lcbrak}[1]{\ensuremath{\left\{#1\right.}}
\providecommand{\rcbrak}[1]{\ensuremath{\left.#1\right\}}}
\theoremstyle{remark}
\newtheorem{rem}{Remark}
\newcommand{\sgn}{\mathop{\mathrm{abs}}}
\providecommand{\abs}[1]{\vert#1\vert}
\providecommand{\res}[1]{\Res\displaylimits_{#1}} 
\providecommand{\norm}[1]{\lVert#1\rVert}
%\providecommand{\norm}[1]{\lVert#1\rVert}
\providecommand{\mtx}[1]{\mathbf{#1}}
\providecommand{\mean}[1]{E[ #1 ]}
\providecommand{\fourier}{\overset{\mathcal{F}}{ \rightleftharpoons}}
%\providecommand{\hilbert}{\overset{\mathcal{H}}{ \rightleftharpoons}}
\providecommand{\system}{\overset{\mathcal{H}}{ \longleftrightarrow}}
	%\newcommand{\solution}[2]{\textbf{Solution:}{#1}}
\newcommand{\solution}{\noindent \textbf{Solution: }}
\newcommand{\cosec}{\,\text{cosec}\,}
\providecommand{\dec}[2]{\ensuremath{\overset{#1}{\underset{#2}{\gtrless}}}}
\newcommand{\myvec}[1]{\ensuremath{\begin{pmatrix}#1\end{pmatrix}}}
\newcommand{\mydet}[1]{\ensuremath{\begin{vmatrix}#1\end{vmatrix}}}
\numberwithin{equation}{subsection}
\makeatletter
\@addtoreset{figure}{problem}
\makeatother
\let\StandardTheFigure\thefigure
\let\vec\mathbf
\renewcommand{\thefigure}{\theproblem}
\def\putbox#1#2#3{\makebox[0in][l]{\makebox[#1][l]{}\raisebox{\baselineskip}[0in][0in]{\raisebox{#2}[0in][0in]{#3}}}}
     \def\rightbox#1{\makebox[0in][r]{#1}}
     \def\centbox#1{\makebox[0in]{#1}}
     \def\topbox#1{\raisebox{-\baselineskip}[0in][0in]{#1}}
     \def\midbox#1{\raisebox{-0.5\baselineskip}[0in][0in]{#1}}
\vspace{3cm}
\title{EE3025 Assignment-1}
\author{Shreekara Raghavan  - EE18BTECH11040}
\maketitle
\newpage
\bigskip
\renewcommand{\thefigure}{\theenumi}
\renewcommand{\thetable}{\theenumi}
Download all python codes from 
\begin{lstlisting}
https://github.com/Shreekara277/IDP/tree/main/codes
\end{lstlisting}

\section{Problem}
    
The command
\begin{lstlisting}
    output_signal = signal.lfilter(b,a,output_signal)
\end{lstlisting}
in Problem 2.3 is executed through following difference equation 
    \begin{align}
    \label{eq:equation1}
        \sum _{m=0}^{M}a\brak{m}y\brak{n-m}=\sum _{k=0}^{N}b\brak{k}x\brak{n-k}
    \end{align}
 where input signal is $x(n)$ and output signal is $y(n)$ with intial values all 0. Replace \textbf{signal.filtfilt} with your own routine and verify.
  \section{Solution}
 Applying Z transform on the equation 1.0.1 we get,
     \begin{align}
        \mathcal {Z}( \sum _{m=0}^{M}a\brak{m}y\brak{n-m} )= \mathcal {Z}( \sum _{k=0}^{N}b\brak{k}x\brak{n-k} )
    \end{align}
     \begin{align}
        \sum _{m=0}^{M} \mathcal {Z} (a\brak{m}y\brak{n-m})= \sum _{k=0}^{N}\mathcal {Z} (b\brak{k}x\brak{n-k} )
    \end{align}
Using the property
  \begin{align}
      {\mathcal {Z}}\{x(n-a)\} = z^{-a}X(z) 
  \end{align}
  where $X(z)$ is the Z- transform of x(n).\bigskip
  \eqref{eq:equation1}  We then get:
\begin{align}
     \sum _{m=0}^{M}a\brak{m} Y\brak{z} z^{-m} =  \sum _{k=0}^{N}b\brak{k}X\brak{z}z^{-k}
\end{align}
\begin{align}
    Y\brak{z} = \frac{\sum _{k=0}^{N}b\brak{k}z^{-k}}{\sum _{m=0}^{M}a\brak{k}z^{-m}}{X\brak{z}}
\end{align}
The numerator and denominator of the butterworth coefficients are b and a respectively which can be calculated.By taking the fft of x(n) and multiplying it with the transfer function,
\begin{align}
    H\brak{z} = \frac{\sum _{k=0}^{N}b\brak{k}z^{-k}}{\sum _{m=0}^{M}a\brak{k}z^{-m}}
\end{align}
We get Y\brak{z}. Taking the ifft of Y\brak{z} will yield us y(n).
\bigskip
Python code for the problem
\begin{lstlisting}
codes/ee18btech11040.py
\end{lstlisting}
 
 Below are the plots which verifies our own routine
 \begin{figure}[!ht]
    \centering
    \includegraphics[width=\columnwidth]{./Figure_1.png}
    \caption{Time response}
    \label{fig:ee18btech11042_1}
\end{figure}
 \begin{figure}[!ht]
    \centering
    \includegraphics[width=\columnwidth]{./Figure_2.png}
    \caption{Frequency response}
    \label{fig:ee18btech11042_2}
\end{figure}
\end{document}